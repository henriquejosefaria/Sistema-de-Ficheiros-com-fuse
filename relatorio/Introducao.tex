
 
\section{Descrição do funcionamento do trabalho desenvolvido}

 \par Neste trabalho começamos por criar a base de dados \textit{fuse} as coleções do mongo onde se guardam os utilizadores\textit{(nome do utilizador no sistema e email)}, códigos por validar\textit{(nome do utilizador do sistema e código)}, um registo de logs de acesso ao sistema \textit{(nome do utilizador no sistema, timestamp da data de acesso e um marcador que diz se o acesso é válido ou se já se tornou inválido)} e um registo de possiveis tentativas de ransomware \textit{(nome do utilizador no sistema e timestamp da data de acesso)}.\newline
 \par Posteriormente procedeu-se á implementação, em Python, do mecanismo de controlo de acessos ao sistema de ficheiros. Para poder acedr ao sistema de ficheiros, o utilizador tem de fornecer um email válido no qual vai receber um código de segurança gerado aleatóriamente e que permanece válido por 1 minuto. Caso o código inserido pelo utilizador esteja errado mas dentro do limite de tempo este é informado que se enganou no código e que ainda tem tempo para voltar a tentar. Caso o código seja inserido após o tempo limite o código é considerado inválido e o utilizador é retornado á página inicial para poder pedir um novo código. Caso o código inserido pelo cliente seja válido e tenha sido inserido dentro do prazo limite é garantido ao utilizador acesso por 2 minutos ao sistema de ficheiros. Após os 2 minutos de acesso garantido, um novo email é enviado ao utilizador para que este possa readquirir acesso ao sistema por mais 2 minutos.\newline
 \par Para que este sistema funcione todos os utilizadores válido têm de estar registados na base de dados na coleção \textit{users}. Sempre que um código é requisitado é adicionado um documento com o userId e o código á coleção \textit{validCodes} que acaba por ser removido da base de dados quando o utilizador se autentica corretamente ou quando o tempo para se autenticar acaba. Ao proceder corretamente á inserção do código secreto o documento usado para guardar o código  e  o nome do utilizador é descartado da base de dados e porcede-se á inserção na coleção \textit{log} um documento que regista o nome do utilizador que acedeu aos ficheiros um timestamp que indica o momento que que acedeu aos mesmos e se o acesso se encontra válido ou se já foi invalidado.\newline
 \par Adicionalmente foi adicionada a funcionalidade de, ao alterar um ficheiro, caso esta alteração seja suspeita de se poder tratar de \textit{ransomware}, o ficheiro é salvo numa diretoria chamada \textit{safe} para podermos recuperar o estado pré-ataque. O cálculo de um possível ataque de \textit{ransomware} é feito pela função \textit{entropia} que é chamada sempre que se usa a função \textit{write} do ficheiro \textit{Passthrough.py}.\newline
 \par Para além disso foi implementada a procura da injeção de malware nos ficheiros usando o clamav. Para isso descarregamos a base de dados de malwares e a biblioteca de python que permite a sua utilização através do método \textit{scan\_stream()}.

\section{Passos para execução do trabalho}

 Para correr o trabalho é necessário ter uma base de dados mongo instalada no computador. São necessários apenas os documentos referentes aos utilizadores no sistema. No nosso caso, os documentos eram os seguintes:\newline


 \par \{ "\_id": ObjectId("5e183e53ad2e22d27a2f1ffb"), "userId": 510, "username" : "henrique",\newline "email" :\texttt{"henriquejosefaria@gmail.com"}\} \newline
 \par \{ "\_id": ObjectId("5e183e53ad2e22d27a2f1ffc"), "userId": 510, "username" : "paulo",\newline "email" :\texttt{"Paulo\_jorge\_000\_\@hotmail.com"}\}\newline



\par\textit{ Note-se que o campo userId resulta da execução do comando "id -u".}\newline


\par De seguida serão precisos realizar os seguintes passos pela ordem em que aparecem:\newline


\begin{itemize}
\item Cria a pasta onde será montado o sistema de ficheiros.\newline
\item Ligar o servidor com o comando python3 webApp.py.\newline
\item Correr o daemon do Antivirus Clamav, isto é, o clamd
\item Proceder á montagem e uso do sistema de ficheiros com o comando: \textit{pyhton3 Passthrough.py \textless DiretoriaOrigem\textgreater \textless DiretoriaMount\textgreater}.\newline


A Diretoria Origem refere-se a diretoria a "copiar", isto é, a fazer a ligação para a diretoria de Mount
A Diretoria Mount é a diretoria onde vai estar o sistema de ficheiros com autentificação\newline
Nota : Todas as bibliotecas necessárias de python podem ser instaladas com o Pip3. O clamav necessita instalar o próprio daemon.
\end{itemize}

\textit{ A curta duração do acesso serve o propósito de mostrar a funcionalidade do sistema de negação de acesso após um tempo estipulado e pré estabelecido, bem como a reaquisição do acesso por parte do utilizador.} 








